\section{Exploring Distributed TensorFlow and Horovod}

Our goal is to match or exceed the performance of current state of the art frameworks. Horovod is a popular distributed SGD implementation which claims near linear scalability to 128 GPUs. They do this by leveraging the ring-allreduce algorithm to synchronize gradients between workers. The algorithm is roughly as follows:

\textbf{Pseudocode.}
\begin{algorithm}
\caption{All-Reduce SGD}
\KwData{Input Data}
\For{all Nodes (simultaneously)} {
\For{all GPUs (simultaneously)}{
    instantiate data\;
    instantiate model replica\;
}
}
Barrier\;

\For{all Nodes (simultaneously)} {
\For{all GPUs (simultaneously)}{
    compute gradient based on data\;
    allreduce(gradient)\;
}
}


\For{all Nodes (simultaneously)} {
\For{all GPUs (simultaneously)}{
    apply gradient to model\;
}
}
\end{algorithm}
\clearpage

Distributed TensorFlow offers a variety of different distributed SGD implementations. For benchmarking we use a synchronized parameter server to reduce across nodes and an averaging scheme within node across GPUs. The algorithm is roughly as follows:

\textbf{Pseudocode.}
\begin{algorithm}
\caption{Synchronous Parameter Server SGD}
\KwData{Input Data, K Nodes}
\For{all Nodes (simultaneously)} {
\For{all GPUs (simultaneously)}{
    instantiate data\;
    instantiate model replica\;
}
}
generate N parameter server shards\;
\For{each shard in the N shards} {
    start parameter server on a different node \;
}
Barrier\;
\For{all Nodes (simultaneously)} {
\For{all GPUs (simultaneously)}{
    compute gradient based on data\;
    allreduce(gradient) on node\;
}
send shard of gradient to corresponding parameter server \;
wait for aggregated gradient \;
apply gradient to model \;
}
\For{all Parameter servers (simultaneously)}{
    \While {number of gradient shards $<$ K\;}{
        wait for gradient shard\;
    }

    aggregate all gradients\;
    broadcast gradients back to workers\;
}
\end{algorithm}

Note that in both cases, the amount of data transferred is constant \textit{per node} and is equal to $2 * N$, where $N$ is the size of the gradient. The total amount of bandwidth consumed scales linearly with the number of nodes.

To understand this, consider ring reduce. For $k$ participants, the gradient buffer is split into $k$ pieces, and each nodes sends and receives $N / k$ bytes over $k$ rounds. In total this means each node sends and receives $2 * N$ bytes.

A similar accounting holds for the parameter server. Each gradient produce must send $N$ bytes to the parameter server, and receive $N$ bytes from the parameter server. If the gradient is sharded into $k$ shards, then the network usage can be balanced evenly across all machines.

The main tradeoff between the ring reduce and shareded parameter server approach then, is the communication pattern itself. The parameter server requires all-to-all network communication, while ring reduce only requires neighboring nodes to communicate. However at the scales we are dealing with (up to a dozen nodes and a hundred GPUs), this is unlikely to be a bottleneck.
